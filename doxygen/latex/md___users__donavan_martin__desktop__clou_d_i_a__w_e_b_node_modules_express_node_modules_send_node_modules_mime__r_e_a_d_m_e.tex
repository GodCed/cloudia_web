Comprehensive M\+I\+M\+E type mapping A\+P\+I based on mime-\/db module.

\subsection*{Install}

Install with \href{http://github.com/isaacs/npm}{\tt npm}\+: \begin{DoxyVerb}npm install mime
\end{DoxyVerb}


\subsection*{Contributing / Testing}

\begin{DoxyVerb}npm run test
\end{DoxyVerb}


\subsection*{Command Line}

\begin{DoxyVerb}mime [path_string]
\end{DoxyVerb}


E.\+g. \begin{DoxyVerb}> mime scripts/jquery.js
application/javascript
\end{DoxyVerb}


\subsection*{A\+P\+I -\/ Queries}

\subsubsection*{mime.\+lookup(path)}

Get the mime type associated with a file, if no mime type is found {\ttfamily application/octet-\/stream} is returned. Performs a case-\/insensitive lookup using the extension in {\ttfamily path} (the substring after the last \textquotesingle{}/\textquotesingle{} or \textquotesingle{}.\textquotesingle{}). E.\+g.


\begin{DoxyCode}
\hyperlink{018__def_8c_a335628f2e9085305224b4f9cc6e95ed5}{var} mime = require(\textcolor{stringliteral}{'mime'});

mime.lookup(\textcolor{stringliteral}{'/path/to/file.txt'});         \textcolor{comment}{// => 'text/plain'}
mime.lookup(\textcolor{stringliteral}{'file.txt'});                  \textcolor{comment}{// => 'text/plain'}
mime.lookup(\textcolor{stringliteral}{'.TXT'});                      \textcolor{comment}{// => 'text/plain'}
mime.lookup(\textcolor{stringliteral}{'htm'});                       \textcolor{comment}{// => 'text/html'}
\end{DoxyCode}


\subsubsection*{mime.\+default\+\_\+type}

Sets the mime type returned when {\ttfamily mime.\+lookup} fails to find the extension searched for. (Default is {\ttfamily application/octet-\/stream}.)

\subsubsection*{mime.\+extension(type)}

Get the default extension for {\ttfamily type}


\begin{DoxyCode}
mime.extension(\textcolor{stringliteral}{'text/html'});                 \textcolor{comment}{// => 'html'}
mime.extension(\textcolor{stringliteral}{'application/octet-stream'});  \textcolor{comment}{// => 'bin'}
\end{DoxyCode}


\subsubsection*{mime.\+charsets.\+lookup()}

Map mime-\/type to charset


\begin{DoxyCode}
mime.charsets.lookup(\textcolor{stringliteral}{'text/plain'});        \textcolor{comment}{// => 'UTF-8'}
\end{DoxyCode}


(The logic for charset lookups is pretty rudimentary. Feel free to suggest improvements.)

\subsection*{A\+P\+I -\/ Defining Custom Types}

Custom type mappings can be added on a per-\/project basis via the following A\+P\+Is.

\subsubsection*{mime.\+define()}

Add custom mime/extension mappings


\begin{DoxyCode}
mime.define(\{
    \textcolor{stringliteral}{'text/x-some-format'}: [\textcolor{stringliteral}{'x-sf'}, \textcolor{stringliteral}{'x-sft'}, \textcolor{stringliteral}{'x-sfml'}],
    \textcolor{stringliteral}{'application/x-my-type'}: [\textcolor{stringliteral}{'x-mt'}, \textcolor{stringliteral}{'x-mtt'}],
    \textcolor{comment}{// etc ...}
\});

mime.lookup(\textcolor{stringliteral}{'x-sft'});                 \textcolor{comment}{// => 'text/x-some-format'}
\end{DoxyCode}


The first entry in the extensions array is returned by {\ttfamily mime.\+extension()}. E.\+g.


\begin{DoxyCode}
mime.extension(\textcolor{stringliteral}{'text/x-some-format'}); \textcolor{comment}{// => 'x-sf'}
\end{DoxyCode}


\subsubsection*{mime.\+load(filepath)}

Load mappings from an Apache \char`\"{}.\+types\char`\"{} format file


\begin{DoxyCode}
mime.load(\textcolor{stringliteral}{'./my\_project.types'});
\end{DoxyCode}
 The .types file format is simple -\/ See the {\ttfamily types} dir for examples. 