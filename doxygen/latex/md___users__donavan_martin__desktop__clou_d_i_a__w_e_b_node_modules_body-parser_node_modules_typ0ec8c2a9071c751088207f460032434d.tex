\href{https://npmjs.org/package/mime-db}{\tt !\mbox{[}N\+P\+M Version\mbox{]}\mbox{[}npm-\/version-\/image\mbox{]}} \href{https://npmjs.org/package/mime-db}{\tt !\mbox{[}N\+P\+M Downloads\mbox{]}\mbox{[}npm-\/downloads-\/image\mbox{]}} \href{http://nodejs.org/download/}{\tt !\mbox{[}Node.\+js Version\mbox{]}\mbox{[}node-\/image\mbox{]}} \href{https://travis-ci.org/jshttp/mime-db}{\tt !\mbox{[}Build Status\mbox{]}\mbox{[}travis-\/image\mbox{]}} \href{https://coveralls.io/r/jshttp/mime-db?branch=master}{\tt !\mbox{[}Coverage Status\mbox{]}\mbox{[}coveralls-\/image\mbox{]}}

This is a database of all mime types. It consists of a single, public J\+S\+O\+N file and does not include any logic, allowing it to remain as un-\/opinionated as possible with an A\+P\+I. It aggregates data from the following sources\+:


\begin{DoxyItemize}
\item \href{http://www.iana.org/assignments/media-types/media-types.xhtml}{\tt http\+://www.\+iana.\+org/assignments/media-\/types/media-\/types.\+xhtml}
\item \href{http://svn.apache.org/repos/asf/httpd/httpd/trunk/docs/conf/mime.types}{\tt http\+://svn.\+apache.\+org/repos/asf/httpd/httpd/trunk/docs/conf/mime.\+types}
\end{DoxyItemize}

\subsection*{Installation}


\begin{DoxyCode}
1 npm install mime-db
\end{DoxyCode}


If you\textquotesingle{}re crazy enough to use this in the browser, you can just grab the J\+S\+O\+N file\+:


\begin{DoxyCode}
1 https://cdn.rawgit.com/jshttp/mime-db/master/db.json
\end{DoxyCode}


\subsection*{Usage}


\begin{DoxyCode}
\hyperlink{018__def_8c_a335628f2e9085305224b4f9cc6e95ed5}{var} db = require(\textcolor{stringliteral}{'mime-db'});

\textcolor{comment}{// grab data on .js files}
\hyperlink{018__def_8c_a335628f2e9085305224b4f9cc6e95ed5}{var} data = db[\textcolor{stringliteral}{'application/javascript'}];
\end{DoxyCode}


\subsection*{Data Structure}

The J\+S\+O\+N file is a map lookup for lowercased mime types. Each mime type has the following properties\+:


\begin{DoxyItemize}
\item {\ttfamily .source} -\/ where the mime type is defined. If not set, it\textquotesingle{}s probably a custom media type.
\begin{DoxyItemize}
\item {\ttfamily apache} -\/ \href{http://svn.apache.org/repos/asf/httpd/httpd/trunk/docs/conf/mime.types}{\tt Apache common media types}
\item {\ttfamily iana} -\/ \href{http://www.iana.org/assignments/media-types/media-types.xhtml}{\tt I\+A\+N\+A-\/defined media types}
\end{DoxyItemize}
\item {\ttfamily .extensions\mbox{[}\mbox{]}} -\/ known extensions associated with this mime type.
\item {\ttfamily .compressible} -\/ whether a file of this type is can be gzipped.
\item {\ttfamily .charset} -\/ the default charset associated with this type, if any.
\end{DoxyItemize}

If unknown, every property could be {\ttfamily undefined}.

\subsection*{Contributing}

To edit the database, only make P\+Rs against {\ttfamily src/custom.\+json} or {\ttfamily src/custom-\/suffix.\+json}.

To update the build, run {\ttfamily npm run update}.

\subsection*{Adding Custom Media Types}

The best way to get new media types included in this library is to register them with the I\+A\+N\+A. The community registration procedure is outlined in \href{http://tools.ietf.org/html/rfc6838#section-5}{\tt R\+F\+C 6838 section 5}. Types registered with the I\+A\+N\+A are automatically pulled into this library. 