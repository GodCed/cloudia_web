\section*{Clou\+D\+I\+A}

\subsection*{Users}

\subsection*{Developpers}

\subsubsection*{Clou\+D\+I\+As\textquotesingle{} members}


\begin{DoxyItemize}
\item 
\end{DoxyItemize}In alphabetical order) Cédric Danick Donavan Julie (C\+E\+O) ... 

\subsubsection*{Commands git (quick help or debug)}


\begin{DoxyPre}
Technique if you do it wrong => see below (allow to recovery to old commit version)
git reset –soft HEAD~1 
git add –A
git commit –m 
git push
git pull
git push\end{DoxyPre}



\begin{DoxyPre}If you delete the branch... rewrite on the branch... we still 
git fsck --full --no-reflogs --unreachable --lost-found
ls -1 .git/lost-found/commit/ | xargs -n 1 git log -n 1 --pretty=oneline
git checkout -b branch-name SHA\end{DoxyPre}



\begin{DoxyPre}*git init*
Commande de base : **git init**\end{DoxyPre}



\begin{DoxyPre}*git clone*
Commande de base : **git clone**\end{DoxyPre}



\begin{DoxyPre}*git add*
Commande de base : **git add**
git add . : ajout de nouveaux fichiers sans suppression
git add -u : ajout de fichiers modifiés et supprimés (sans nouveau ajout) 
git add -A : ajout des fichiers (raccourci de git add .; git add –u)\end{DoxyPre}



\begin{DoxyPre}*git branch*
Commande de base : **git branch**
git branch <branch> : ajout d’une branche
git branch –d <branch> : ne plus regarder cette branche 
git branch –D <branch> : force la suppression de la branche
git branch –m <branch> : renomme la branche sur laquelle je suis
git branch –a : liste les branches locales et les branches suivi à distance\end{DoxyPre}



\begin{DoxyPre}*git checkout*
Commande de base : **git checkout**
git checkout <existing-branch> : regarde si la branche a été créer avec git branch et change
git checkout –b <new-branch> <branch> : créer une nouvelle branche si n’existe pas déjà
git\end{DoxyPre}



\begin{DoxyPre}*git merge*
Commande de base : **git merge**
git merge <branch> : merge la branche spécifier à celle actuelle
git merge –no-ff <branch> : générer un merge commit (on peut avoir une trace)
git\end{DoxyPre}



\begin{DoxyPre}*git pull*
Commande de base : **git pull**
git pull --rebase origin master : option --rebase déplace tous les commit à la branche spéficier
git status
git\end{DoxyPre}



\begin{DoxyPre}*git rebase*
Commande de base : **git rebase**
git rebase --continue
git rebase : pour revenir à la dernière exécution de la commande git pull –rebase\end{DoxyPre}



\begin{DoxyPre}*git push*
Commande de base : **git push**
git push origin <branch> : ...
git push origin <branch> -f : ... 
git push -u origin <branch> : ...
\end{DoxyPre}


\subsubsection*{D\+B\+\_\+setting.\+php}

This is a file ...

\subsection*{Server configuration}

Danick is going to do that soon! 