cookie is a basic cookie parser and serializer. It doesn\textquotesingle{}t make assumptions about how you are going to deal with your cookies. It basically just provides a way to read and write the H\+T\+T\+P cookie headers.

See \href{http://tools.ietf.org/html/rfc6265}{\tt R\+F\+C6265} for details about the http header for cookies.

\subsection*{how?}


\begin{DoxyCode}
1 npm install cookie
\end{DoxyCode}



\begin{DoxyCode}
1 var cookie = require('cookie');
2 
3 var hdr = cookie.serialize('foo', 'bar');
4 // hdr = 'foo=bar';
5 
6 var cookies = cookie.parse('foo=bar; cat=meow; dog=ruff');
7 // cookies = \{ foo: 'bar', cat: 'meow', dog: 'ruff' \};
\end{DoxyCode}


\subsection*{more}

The serialize function takes a third parameter, an object, to set cookie options. See the R\+F\+C for valid values.

\subsubsection*{path}

\begin{quote}
cookie path \end{quote}


\subsubsection*{expires}

\begin{quote}
absolute expiration date for the cookie (Date object) \end{quote}


\subsubsection*{max\+Age}

\begin{quote}
relative max age of the cookie from when the client receives it (seconds) \end{quote}


\subsubsection*{domain}

\begin{quote}
domain for the cookie \end{quote}


\subsubsection*{secure}

\begin{quote}
true or false \end{quote}


\subsubsection*{http\+Only}

\begin{quote}
true or false\end{quote}
