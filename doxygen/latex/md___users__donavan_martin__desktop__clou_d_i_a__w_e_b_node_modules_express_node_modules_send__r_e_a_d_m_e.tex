\href{https://npmjs.org/package/send}{\tt !\mbox{[}N\+P\+M Version\mbox{]}\mbox{[}npm-\/image\mbox{]}} \href{https://npmjs.org/package/send}{\tt !\mbox{[}N\+P\+M Downloads\mbox{]}\mbox{[}downloads-\/image\mbox{]}} \href{https://travis-ci.org/pillarjs/send}{\tt !\mbox{[}Linux Build\mbox{]}\mbox{[}travis-\/image\mbox{]}} \href{https://ci.appveyor.com/project/dougwilson/send}{\tt !\mbox{[}Windows Build\mbox{]}\mbox{[}appveyor-\/image\mbox{]}} \href{https://coveralls.io/r/pillarjs/send?branch=master}{\tt !\mbox{[}Test Coverage\mbox{]}\mbox{[}coveralls-\/image\mbox{]}} \href{https://www.gratipay.com/dougwilson/}{\tt !\mbox{[}Gratipay\mbox{]}\mbox{[}gratipay-\/image\mbox{]}}

Send is a library for streaming files from the file system as a http response supporting partial responses (Ranges), conditional-\/\+G\+E\+T negotiation, high test coverage, and granular events which may be leveraged to take appropriate actions in your application or framework.

Looking to serve up entire folders mapped to U\+R\+Ls? Try \href{https://www.npmjs.org/package/serve-static}{\tt serve-\/static}.

\subsection*{Installation}


\begin{DoxyCode}
1 $ npm install send
\end{DoxyCode}


\subsection*{A\+P\+I}


\begin{DoxyCode}
\hyperlink{018__def_8c_a335628f2e9085305224b4f9cc6e95ed5}{var} send = require(\textcolor{stringliteral}{'send'})
\end{DoxyCode}


\subsubsection*{send(req, path, \mbox{[}options\mbox{]})}

Create a new {\ttfamily Send\+Stream} for the given path to send to a {\ttfamily res}. The {\ttfamily req} is the Node.\+js H\+T\+T\+P request and the {\ttfamily path} is a urlencoded path to send (urlencoded, not the actual file-\/system path).

\paragraph*{Options}

\subparagraph*{dotfiles}

Set how \char`\"{}dotfiles\char`\"{} are treated when encountered. \hyperlink{class_a}{A} dotfile is a file or directory that begins with a dot (\char`\"{}.\char`\"{}). Note this check is done on the path itself without checking if the path actually exists on the disk. If {\ttfamily root} is specified, only the dotfiles above the root are checked (i.\+e. the root itself can be within a dotfile when when set to \char`\"{}deny\char`\"{}).


\begin{DoxyItemize}
\item {\ttfamily \textquotesingle{}allow\textquotesingle{}} No special treatment for dotfiles.
\item {\ttfamily \textquotesingle{}deny\textquotesingle{}} Send a 403 for any request for a dotfile.
\item {\ttfamily \textquotesingle{}ignore\textquotesingle{}} Pretend like the dotfile does not exist and 404.
\end{DoxyItemize}

The default value is {\itshape similar} to {\ttfamily \textquotesingle{}ignore\textquotesingle{}}, with the exception that this default will not ignore the files within a directory that begins with a dot, for backward-\/compatibility.

\subparagraph*{etag}

Enable or disable etag generation, defaults to true.

\subparagraph*{extensions}

If a given file doesn\textquotesingle{}t exist, try appending one of the given extensions, in the given order. By default, this is disabled (set to {\ttfamily false}). An example value that will serve extension-\/less H\+T\+M\+L files\+: `\mbox{[}\textquotesingle{}html\textquotesingle{}, \textquotesingle{}htm\textquotesingle{}\mbox{]}`. This is skipped if the requested file already has an extension.

\subparagraph*{index}

By default send supports \char`\"{}index.\+html\char`\"{} files, to disable this set {\ttfamily false} or to supply a new index pass a string or an array in preferred order.

\subparagraph*{last\+Modified}

Enable or disable {\ttfamily Last-\/\+Modified} header, defaults to true. Uses the file system\textquotesingle{}s last modified value.

\subparagraph*{max\+Age}

Provide a max-\/age in milliseconds for http caching, defaults to 0. This can also be a string accepted by the \href{https://www.npmjs.org/package/ms#readme}{\tt ms} module.

\subparagraph*{root}

Serve files relative to {\ttfamily path}.

\subsubsection*{Events}

The {\ttfamily Send\+Stream} is an event emitter and will emit the following events\+:


\begin{DoxyItemize}
\item {\ttfamily error} an error occurred {\ttfamily (err)}
\item {\ttfamily directory} a directory was requested
\item {\ttfamily file} a file was requested {\ttfamily (path, stat)}
\item {\ttfamily headers} the headers are about to be set on a file {\ttfamily (res, path, stat)}
\item {\ttfamily stream} file streaming has started {\ttfamily (stream)}
\item {\ttfamily end} streaming has completed
\end{DoxyItemize}

\subsubsection*{.pipe}

The {\ttfamily pipe} method is used to pipe the response into the Node.\+js H\+T\+T\+P response object, typically {\ttfamily send(req, path, options).pipe(res)}.

\subsection*{Error-\/handling}

By default when no {\ttfamily error} listeners are present an automatic response will be made, otherwise you have full control over the response, aka you may show a 5xx page etc.

\subsection*{Caching}

It does {\itshape not} perform internal caching, you should use a reverse proxy cache such as Varnish for this, or those fancy things called C\+D\+Ns. If your application is small enough that it would benefit from single-\/node memory caching, it\textquotesingle{}s small enough that it does not need caching at all ;).

\subsection*{Debugging}

To enable {\ttfamily \hyperlink{addon_2doxmlparser_2src_2debug_8cpp_a0a777024bcc965d6200d9599eb187cd9}{debug()}} instrumentation output export {\bfseries D\+E\+B\+U\+G}\+:


\begin{DoxyCode}
1 $ DEBUG=send node app
\end{DoxyCode}


\subsection*{Running tests}


\begin{DoxyCode}
1 $ npm install
2 $ npm test
\end{DoxyCode}


\subsection*{Examples}

\subsubsection*{Small example}


\begin{DoxyCode}
\hyperlink{018__def_8c_a335628f2e9085305224b4f9cc6e95ed5}{var} http = require(\textcolor{stringliteral}{'http'});
\hyperlink{018__def_8c_a335628f2e9085305224b4f9cc6e95ed5}{var} send = require(\textcolor{stringliteral}{'send'});

\hyperlink{018__def_8c_a335628f2e9085305224b4f9cc6e95ed5}{var} app = http.createServer(\textcolor{keyword}{function}(req, res)\{
  send(req, req.url).pipe(res);
\}).listen(3000);
\end{DoxyCode}


Serving from a root directory with custom error-\/handling\+:


\begin{DoxyCode}
\hyperlink{018__def_8c_a335628f2e9085305224b4f9cc6e95ed5}{var} http = require(\textcolor{stringliteral}{'http'});
\hyperlink{018__def_8c_a335628f2e9085305224b4f9cc6e95ed5}{var} send = require(\textcolor{stringliteral}{'send'});
\hyperlink{018__def_8c_a335628f2e9085305224b4f9cc6e95ed5}{var} url = require(\textcolor{stringliteral}{'url'});

\hyperlink{018__def_8c_a335628f2e9085305224b4f9cc6e95ed5}{var} app = http.createServer(\textcolor{keyword}{function}(req, res)\{
  \textcolor{comment}{// your custom error-handling logic:}
  \textcolor{keyword}{function} error(err) \{
    res.statusCode = err.status || 500;
    res.end(err.message);
  \}

  \textcolor{comment}{// your custom headers}
  \textcolor{keyword}{function} headers(res, path, stat) \{
    \textcolor{comment}{// serve all files for download}
    res.setHeader(\textcolor{stringliteral}{'Content-Disposition'}, \textcolor{stringliteral}{'attachment'});
  \}

  \textcolor{comment}{// your custom directory handling logic:}
  \textcolor{keyword}{function} redirect() \{
    res.statusCode = 301;
    res.setHeader(\textcolor{stringliteral}{'Location'}, req.url + \textcolor{charliteral}{'/'});
    res.end(\textcolor{stringliteral}{'Redirecting to '} + req.url + \textcolor{charliteral}{'/'});
  \}

  \textcolor{comment}{// transfer arbitrary files from within}
  \textcolor{comment}{// /www/example.com/public/*}
  send(req, url.parse(req.url).pathname, \{root: \textcolor{stringliteral}{'/www/example.com/public'}\})
  .on(\textcolor{stringliteral}{'error'}, error)
  .on(\textcolor{stringliteral}{'directory'}, redirect)
  .on(\textcolor{stringliteral}{'headers'}, headers)
  .pipe(res);
\}).listen(3000);
\end{DoxyCode}


\subsection*{License}

\mbox{[}M\+I\+T\mbox{]}(L\+I\+C\+E\+N\+S\+E) 